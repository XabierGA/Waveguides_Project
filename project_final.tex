\documentclass[a4paper,12pt]{article} 
\usepackage[english]{babel}
\usepackage[utf8]{inputenc}
\usepackage[T1]{fontenc} % Permite cambiar la fuente por defecto.
\usepackage{graphicx}    % Permite implementar imágenes.
\usepackage{color}       % Permite el uso de colores.
\usepackage{anysize}     % Permite modificar el tamaño de los márgenes.
\usepackage{multicol}    % Permite escribir a doble, triple...columna.
\usepackage{bm}         
\usepackage{textcomp}    
\usepackage{eurosym}     
\usepackage{amsthm} 
\usepackage{amsmath}     
\usepackage{amsmath,amsfonts} 
\usepackage{lineno} 
\usepackage{float}
\usepackage{booktabs}     
\usepackage{fancyhdr}
\usepackage{longtable}
\usepackage[makeroom]{cancel}

\marginsize{2.5cm}{1.5cm}{1.5cm}{1.5cm} % MÁRGENES: Izq, Der, Sup, Inf.
\parindent=0mm                        % Sangría por defecto. 
\parskip=3mm                          % Espacio entre párrafos por defecto.
\renewcommand{\baselinestretch}{1}    % Interlineado.
\newcommand{\es}{\hspace{0.15cm}}
\newcommand{\vect}[1]{\boldsymbol{#1}}
\pagestyle{fancy}
\fancyhf{}
\rhead{Xabier G. Andrade}
\lhead{Waveguides}
\rfoot{Classical Electrodynamics}
\lfoot{Project}



\title{Classical Electrodynamics Project. Waveguides}
\author{Xabier García Andrade}
\date{10th of October 2018} 

\begin{document}

\maketitle
\tableofcontents

\newpage

\section{Introduction}

Wave Guides A resonant cavity consists of the empty space between two perfectly
conducting, concentric spherical shells, the smaller having an outer radius a and the
larger an inner radius b. As shown in Section 8.9, the azimuthal magnetic field has a
radial dependence given by spherical Bessel functions, $j_l(kr)$ and $n_l(kr)$, where $k =
w/c$.

\section{Exercise a}

\textbf{Write down the transcendental equation for the characteristic frequencies of the
cavity for arbitrary l.}

If we are only interested in the lowest frequencies, we can focus our attention on the TM modes, with only tangential magnetic fields. Considering our spherical symmetry, TM modes notation refers to the absence of radial magnetic field components. Also, we will consider that the fields are independent of the azimuthal angle $ \phi$. In spherical harmonics, the dominant term is $l$ instead of $m$, so it is a suitable consideration. Since $B_r = 0$ and $\vec{B}$ does not depend on $\phi$ , from the divergence equation we obtain: 

\begin{equation}
\nabla \cdot \vect{B} =\frac{1}{r^2} \cancelto{0}{\frac{r^2 \cdot \partial B_{r}}{\partial r}} + \frac{1}{r sin \theta} \cancelto{0}{\frac{\partial B_{\phi}}{\partial \phi}} + \frac{1}{r sin \theta} \frac{\partial B_{\theta} sin \theta}{\partial \theta} = 0
\label{eq:bdiver}
\end{equation}

From this equation we obtain that if the fields are finite at $\theta = 0$, then the only term that would survive is $B_{\phi}$. From Faraday law it follows that: 

\begin{equation}
\nabla \times \vect{E} = -\frac{\partial \vect{B}}{\partial t} = \frac{\partial{B_{\phi}}}{{\partial t}} \hat{\phi}
\label{eq:gauss}
\end{equation}

With this we can conclude that $E_{\phi}$ also has to vanish. In summary, our problem would only consist of finding $E_r$ , $E_{\theta}$ and $B_{\phi}$. We still have to use the other curl equations, which will bring (assuming a time dependence of $e^{-i \omega t}$ and taking permeabilities as unity): 

\begin{equation}
\frac{\omega ^2}{c^2}\vect{B} - \nabla \times \nabla \times \vect{B}  = 0
\label{eq:curl}
\end{equation}

We can write down the equation in terms of the $\phi$ component: 

\begin{equation}
\frac{\omega ^2}{c^2}(r B_{\phi}) + \frac{\partial ^2 }{\partial r^2}(rB_{\phi}) + \frac{1}{r^2}\frac{\partial}{\partial \theta} \left[\frac{1}{sin \theta} \frac{\partial}{\partial \theta} (sin \theta rB_{\phi}) \right] = 0 
\label{eq:phicomp}
\end{equation}

Rewriting the angular part: 

\begin{equation}
\frac{\partial}{\partial \theta} \left[ \frac{1}{\sin \theta} \frac{\partial}{\partial} (sin \theta rB_{\phi}) \right] = \frac{1}{sin \theta} \frac{\partial}{\partial \theta} \left(sin \theta \frac{\partial (rB_{\phi})}{\partial \theta} \right) - \frac{r B_{\theta}}{sin^2 \theta}
\label{eq:angpart}
\end{equation}

We can compare the previous equation with the following one: 

\begin{equation}
\frac{1}{sin \theta} \frac{d}{d \theta} \left(sin \theta \frac{dP}{d \theta} \right) + \left[l(l+1) - \frac{m^2}{sin^2 \theta} \right]P = 0
\label{eq:comp}
\end{equation}

Thus we can easily see that the solution to equation (\ref{eq:angpart}) is given by the associated Legendre polynomials with $m = \pm 1$. We then try a solution of the form: 

\begin{equation}
B_{\phi} (r , \theta) = \frac{u_{l}(r)}{r} P^1_{l}(cos \theta)
\label{eq:sol}
\end{equation}

We then substitute directly into (\ref{eq:phicomp}): 

\begin{equation}
\frac{d^2 u_{l}(r)}{r^2} + \left[\frac{\omega^2}{c^2} - \frac{l(l+1)}{r^2} \right] u_l(r) = 0
\label{eq:freqsol}
\end{equation}

The solutions of this equation can be expressed in terms of spherical Bessel functions. First we need to find the boundary conditions for $u_l (r)$ : 

From (\ref{eq:sol}) we can obtain radial and tangential electric fields:

$$E_r = \frac{ic^2}{\omega r sin \theta}\frac{\partial }{\partial B_{\phi}} =- \frac{ic^2}{\omega r} l (l+1) \frac{u_l (r)}{r} P_l (cos \theta)$$

$$E_{\theta} = - \frac{ic^2}{\omega r} \frac{\partial}{\partial r} (r B_{\phi}) = - \frac{ic^2}{\omega r} \frac{\partial u_l (r)}{\partial r} P_l (cos \theta)$$

Since $E_{\theta}$ must vanish at the surface of both shells, we can obtain the boundary condition for $u_l r$: 

\begin{equation}
\frac{d u_l (r)}{dr} = 0 \hspace{2 cm} for \es r = a \es and \es r =b
\label{eq:boundarycond}
\end{equation}

Now, we can write the solutions as follows: 

\begin{equation}
u_l (r) = r A_l j_l(kr) + r B_l n_l(kr) \hspace{1.5cm} with \hspace{0.5cm} \frac{du_{l}(r)}{dr} = 0 \hspace{1cm} at \es r=a  \es and  \es r=b
\label{eq:sphericalbessel}
\end{equation}

Now we need to differentiate with respect to r to obtain the trascendental equations that characteristic frequencies satisfy: 

\begin{equation}
\frac{d u_l (r)}{dr} = A_l j_l (kr) + r A_l j_l \prime (kr) + B_l n_l(kr) + r B_l n_l \prime (kr)
\label{eq:boundarycharac}
\end{equation} 

By using (\ref{eq:boundarycond}) we end up with the following system of equations: 

\begin{equation}
    \begin{cases}
    A_l[j_l(ka) + ka j_l \prime(ka)] + B_l[n_l(ka) + ka n_l \prime (ka)] = 0\\
    A_l[j_l(kb) + kb j_l \prime(kb)] + B_l[n_l(kb) + kb n_l \prime (kb)] = 0.
  \end{cases}
\end{equation}

We can transform this into a matrix equation: 

\begin{equation}
\begin{bmatrix}     j_l(ka) + ka j_l \prime(ka) & n_l(ka) + ka n_l \prime (ka) \\
    j_l(kb) + kb j_l \prime (kb) & n_l(kb) + kb n_l \prime (kb) \end{bmatrix}
 \begin{bmatrix}
  A_l \\
  B_l 
 \end{bmatrix}
 =
 \begin{bmatrix}
   0 \\
   0
   \end{bmatrix}
\label{eq:matrix}
\end{equation}



Which from Kramer´s theorem we can deduce that non-trivial solutions will happen only when the range of the matrix is less than 2. Then we need the determinant to vanish: 

\begin{equation}
[j_l(ka) + ka j_l \prime(ka)] \cdot [n_l(kb) + kb n_l \prime (kb)] - [n_l(ka) + ka n_l \prime (ka)]\cdot [j_l(kb) + kb j_l \prime (kb)] = 0
\label{eq:determina}
\end{equation} 

Which can be better written as the following quotient, known as the trascendental equation: 
\begin{equation}
\frac{j_l(ka) + ka j_l \prime(ka)}{n_l(ka) + ka n_l \prime (ka)} = \frac{n_l(kb) + kb n_l \prime (kb)}{j_l(kb) + kb j_l \prime (kb)}
\label{eq:quot}
\end{equation}


\newpage 

\section{Exercise b}

\textbf{For l = 1 use the explicit forms of the spherical Bessel functions to show that the
characteristic frequencies are given by}

$$\frac{tankh}{kh} = \frac{(k^2 + \frac{1}{ab})}{k^2 + ab(k^2 - \frac{1}{a^2})(k^2 - \frac{1}{b^2})}$$

\textbf{where $h = b -a.$}

Now we have to consider $l = 1$ to obtain the exact form of the spherical Bessel functions. First we consider spherical bessel functions of first kind: 

\begin{equation}
j_1 (z) = \frac{sin z}{z^2} - \frac{cos z}{z} = \frac{sinz -zcosz}{z^2}
\end{equation}

Second kind:

\begin{equation}
n_1 (z) = - \frac{cos z}{z^2} - \frac{sin z}{z} = \frac{-cosz - zsinz}{z^2}
\end{equation}

Their respective derivatives: 

\begin{equation}
j_1 \prime (z) = \frac{z^2 cos z - 2zsinz}{z^4} - \frac{-z sinz - cosz}{z^2} = \frac{z^2sinz + 2z cosz - 2sinz}{z^3}
\end{equation}

\begin{equation}
n_1 \prime (z) = \frac{z^2sin(z)+2z cos(z)}{z^4} + \frac{-zcosz - sinz}{z^2} = \frac{-z^2cosz + 2z sinz +2cosz}{z^3}
\end{equation}

Replacing each one in the transcendental equation we obtain:
\tiny
\begin{equation}
\frac{kasin(ka) -k^2a^2cos(za) + a^3k^3sin(ka)+2k^2a^2cos(ka)-2kasin(ka)}{-kacos(ka)-k^2a^2sin(ka) - k^3a^3cos(ka)+2k^2a^2sin(ka)+2kacos(ka)} = \frac{kbsin(kb) -k^2b^2cos(zb) + b^3k^3sin(kb)+2k^2b^2cos(kb)-2kbsin(kb)}{-kbcos(kb)-k^2b^2sin(kb) - k^3b^3cos(kb)+2k^2b^2sin(kb)+2kbcos(kb)}
\end{equation}
\normalsize
Simplifying the equation:
\begin{equation}
\frac{(a^2k^2 -1)sin(ka)+kacos(ka)}{(-k^2a^2 + 1)cos(ka)+kasin(ka)} = \frac{(b^2k^2 -1)sin(kb)+kbcos(kb)}{(-k^2b^2 + 1)cos(kb)+kbsin(kb)}
\label{eq:transcsimp}
\end{equation}
Now we can continue operating:

\tiny
\begin{equation}
\begin{split}
-(a^2k^2-1)(b^2k^2-1)sin(ka)cos(kb)-ka(b^2k^2-1)cos(kb)cos(ka)+kb(a^2k^2-1)sin(kb)sin(ka)+k^2bacos(ka)sin(kb) =\\ -(b^2k^2-1)(a^2k^2-1)cos(ka)sin(kb) + ka(b^2k^2-1)sin(ka)sin(kb) -kb(a^2k^2 -1)cos(ka)cos(kb)+k^2abcos(kb)sin(ka)
\end{split}
\end{equation}
\normalsize
Grouping similar terms together: 

\begin{small}
\begin{equation}
\begin{split}
-(a^2k^2-1)(b^2k^2-1)[sin(ka)cos(kb)-cos(ka)sin(kb)] + k^2ab[sin(kb)cos(ka) - sin(ka)cos(kb)] = \\ ka(b^2k^2-1)[cos(kb)cos(ka) + sin(ka)sin(kb)] - kb(a^2k^2 -1 )[cos(ka)cos(kb) + sin(ka)sin(kb)]
\end{split}
\end{equation}
\end{small}

Now using the following trigonometric relations: 

\begin{equation}
    \begin{cases}
    sin(\alpha - \beta) = sin\alpha cos \beta - cos \alpha sin \beta\\
    cos(\alpha - \beta) = cos \alpha cos \beta + sin \beta sin \alpha
  \end{cases}
\end{equation}

We get to the following expression: 

\begin{small}
\begin{equation}
\begin{split}
-(a^2k^2-1)(b^2k^2-1)sin(ka-kb) + k^2ab sin(kb - ka) = \\ ka(b^2k^2-1)cos(kb - ka) - kb(a^2k^2 -1 )cos (ka - kb)
\end{split}
\end{equation}
\end{small}

Now using the fact that $sin(- \alpha) = - sin (\alpha) $ and $cos(- \alpha) ) = cos(\alpha)$ and grouping terms: 

\begin{equation}
\left[(a^2k^2-1)(b^2k^2-1) + k^2ab \right]sin(kb-ka) = \left[ka(b^2k^2-1) - kb(a^2k^2 -1) \right]cos(kb - ka)
\end{equation}

Now defining $h = b-a$ we get: 

\begin{equation}
tan(hk) = \frac{ka(b^2k^2-1) - kb(a^2k^2 -1)}{(a^2k^2-1)(b^2k^2-1) + k^2ab} = \frac{k^3ba(b-a) + k(b-a)}{a^2b^2k^4 -a^2k^2 -b^2k^2 +1 +k^2ab}
\end{equation}

\begin{equation}
tan(hk) =kh \frac{k^2ba +1}{abk^2+a^2b^2k^4 -a^2k^2-b^2k^2+1} 
\end{equation}

Then we get to the final solution: 

\begin{equation}
\frac{tan(hk)}{hk} = \frac{k^2 + \frac{1}{ab}}{k^2 + ab\left(k^2 - \frac{1}{a^2} \right)\left(k^2 - \frac{1}{b^2} \right)}
\label{eq:finalsol}
\end{equation}

\section{Exercise c}

\textbf{Write down the transcendental equation for the characteristic frequencies of the
cavity for arbitrary l.}

Now we must consider the following approximation : $ h/a << 1$

Rewriting (\ref{eq:finalsol}) in terms of $h/a$: 

\begin{equation}
\frac{k^2 + \frac{1}{a^2(1+\frac{h}{a})}}{k^2+a^2(1+h/a)(k^2-\frac{1}{a^2}) \left( k^2 - \frac{1}{a^2(1 + \frac{h}{a})^2}\right)}
\label{eq:ran}
\end{equation}

Now we must perform a Taylor expansion both in the numerator and the denominator. We will only consider first order terms because the following ones are neglible. 

\begin{equation}
\frac{1}{1+\frac{h}{a}} \approx 1 - \frac{h}{a}
\label{eq:taylor}
\end{equation}
\begin{equation}
\frac{1}{(1+\frac{h}{a})^2} \approx	 1 - 2  \frac{h}{a}
\label{eq:taylorsquare} 
\end{equation}

Plugging this result in (\ref{eq:ran}): 

\begin{equation}
\frac{tan kh}{kh} = \frac{k^2 + \frac{1}{a^2} - \frac{1}{a^2} \frac{h}{a}}{k^2 + (1+ \frac{h}{a})(k^2a^2 - 1)(k^2 - \frac{1}{a^2} + \frac{2}{a^2}\frac{h}{a})}
\end{equation}

Multiplicating factors in the denominator: 

\begin{equation}
\frac{tan kh}{kh} = \frac{k^2 + \frac{1}{a^2} - \frac{1}{a^2} \frac{h}{a}}{k^2 + k^4a^2 -k^2 + k^2 \frac{h}{a} + k^4ah - k^2 + \frac{1}{a^2}
- \frac{1}{a^2} \frac{h}{a} - k^2 \frac{h}{a}}
\end{equation}

\begin{equation}
\frac{tankh}{kh}\frac{k^2 + \frac{1}{a^2} - \frac{1}{a^2} \frac{h}{a}}{k^2(k^2a^2-1) + \frac{h}{a}(k^4a^2 -\frac{1}{a^2})}
\end{equation}

Now we can also expand the left-hand term considering again $h/a <<1$: 

\begin{equation}
\frac{tan kh}{kh} \approx 1 + \mathcal{O}((h/a)^2)
\end{equation}

Where again we can disregard second order terms. Now writing it all together: 

\begin{equation}
k^2(k^2a^2-1) \frac{1}{a^2} + \frac{h}{a} \left(k^4a^2 - \frac{1}{a^2} \right) = k^2 + \frac{1}{a^2} - \frac{1}{a^2} \frac{h}{a}	
\end{equation}

Doing further manipulations we arrive to:

\begin{equation}
k^2 = \frac{2}{a(a+h)}
\end{equation}

We can relate $k$ to the frequency by means of the speed of light $c$ ($\omega = ck$): 

\begin{equation}
\omega ^2 = c^2 k^2 = 2c^2 \frac{1}{a(a+h)} = \frac{2 c^2}{a^2} \frac{1}{(1+\frac{h}{a})}
\end{equation}

Using (\ref{eq:taylor}) to expand the denominator as we did before: 

\begin{equation}
\omega^2 = \frac{2c^2}{a^2} (1 - \frac{h}{a})
\end{equation}

\begin{equation}
\omega = \frac{\sqrt{2} c}{a} \sqrt{1 - \frac{h}{a}}
\end{equation} 	

We need to perform another Taylor expansion to retrieve the result from Jackson: 

\begin{equation}
\sqrt{1 - \frac{h}{a}} = 1 - \frac{h}{2a}
\end{equation}

Then , the value of the frequency would be: 

\begin{equation} 
\omega = \sqrt{2} \frac{c}{a} - \frac{\sqrt{2}}{2} \frac{h}{a}
\label{eq:lab}
\end{equation}

We can compare this to the result given in Jackson´s equation (8.105): 

\begin{equation} 
\omega_l \approx   \sqrt{l (l+1)} \frac{c}{a}
\end{equation}

For $l=1$ , we obtain $\omega_1 = \sqrt{2} \frac{c}{a}$. Thus, we can conclude that the secon term in (\ref{eq:lab}) is the first order correction.



\section{Bibliography}

\begin{itemize}

\item Jackson, John David. Classical Electrodynamics, 3rd edition, 1998.
\tiny
\item WolframMathWorld \textbf{http://mathworld.wolfram.com/SphericalBesselFunctionoftheFirstKind.html}
\item WolframMathWorld \textbf{http://mathworld.wolfram.com/SphericalBesselFunctionoftheSecondKind.html}
\end{itemize}


\end{document} 